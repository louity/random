\section*{introduction}

Pendant la présentation des projets par les entreprises, nous avons été séduits par le projet de prédiction de la qualité de l'air à l'échelle de la rue de Plume Labs.
En effet, le problème en question est un problème de physique, et l'on peut espérer s'inspirer de son intuition pour proposer des solutions.
De plus, l'application, la prédiction de la pollution à l'echelle de la rue, est intéressante.

Cependant, quand nous avons commencé à étudier les données, nous nous sommmes rendus compte que tous la plupart des paramètre physiques n'étaient pas inclus dans les données fournies, ce qui empêchait d'avoir une approche physique du problème et qui remettait sérieusement en cause l'ambition affichée: prédire la qualité de l'ai à l'échelle de la rue.
Nous avons contacté Plume Labs pour en savoir plus, et ils nous ont répondu que cela était volontaire, le but réel du projet étant de "prédire la pollution à des points précis à partir d'informations statiques sur ces points et d'informations structurelles extrautes sur les séries temporelles".

C'est donc à ce problème que nous nous sommes attaqués, problème très intéressant, mais aussi très général pour lequel est difficile d'espérer avoir des prédictions précises.

