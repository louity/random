\section{Techniques existantes}

Dans la description du challenge, Plume Labs donne les références de deux articles qui décrivent et implémentent un modèle de régression de la pollution.
Le premier \cite{} traite exclusivement le cas de la pollution en $NO_2$, qui est prédité uniquement à partir de variables statiques.
Elles sont beaucoup plus précises que celles que nous avons à disposition:
\begin{itemize}
  \item 4 types de routes sont distingués
  \item on a les valeurs u traffic routier moyen dans des zones de différents rayons
  \item l'altitude est donnée
  \item les batiments sont classés par taille
\end{itemize}

De plus, les données d'entrainement sont beaucoup plus conséquentes: pour une seule ville, il y a 25 points pour lequels la pollution est connue et nous n'en avons que 3 par ville.

Le second \cite{} traite exclusivement le cas des particules ultrafines (diamètre inférieur à 0.1 micromètres), auxquelles n'appartiennent pas les particules $PM_{10}$ et $PM_{2.5}$ que nous étudions.
Cette fois ci, les données sont constittuées de données statiques et de données météorologique, et là encore, les données statiques sont beaucoup plus précises que celles que nous avons.
